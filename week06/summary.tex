
\twocolumn[{%
\begin{center}
  {\LARGE \textbf{\textsf{Top Quark Seminar 6}}} \\
  \vspace{1em}
  {\Large \textbf{\textsf{Igor Babuschkin}}} \\
  \vspace{1em}
  {\large \textbf{\textsf{10th February 2015}}}
  \section*{Summary of \enquote{Measurement of Top Quark Polarization in Top-Antitop Events from Proton-Proton Collisions at $\sqrt{s}=\SI{7}{TeV}$ Using the ATLAS Detector}}
\end{center}
}]

\noindent
The paper\cite{polarization} documents the first measurement of the top quark polarization in $t\overline{t}$ events, performed using data taken by the ATLAS collaboration during 2011 with an integrated luminosity of \SI{4.7}{\text{fb}^{-1}} at $\sqrt{s}=\SI{7}{TeV}$.

The Standard Model predicts that top quarks from $t\overline{t}$ pairs produced in proton-proton collisions at the LHC should have negligible polarization.

A measurement of the top quark polarization is interesting, because several models of Physics beyond the Standard Model could generate a top quark polarization.
In particular this is the case for models that could explain the anomalous results for the top quark production forward-backward asymmetries seen by CDF and DØ.

The angular component of the $t\overline{t}$ production cross section can be parameterized as in \eqref{parametrization}, where $\theta_{\{1,2\}}$ are the polar angles of the $t$ and $\overline{t}$ decay products,  $P_{\{1,2\}}$ are the top quark polarizations, $C$ is the top quark spin correlation and $\alpha_{\{1,2\}}$ is the spin-analyzing power of the respective final state.
The spin-analyzing power is a measure of the correlation between top quark and final state spins.

\begin{equation}
  \begin{split}
    \frac{1}{\sigma} \frac{d\sigma}{d\mathrm{cos}(\theta_1)d\mathrm{cos}(\theta_2)} = \frac{1}{4}(1 + \alpha_1 P_1 \mathrm{cos}(\theta_1) + \\ \alpha_2 P_2 \mathrm{cos}{\theta_2} - C \mathrm{cos}(\theta_1)\mathrm{cos}(\theta_2))
  \end{split}
  \label{parametrization}
\end{equation}

The analysis is based on five different possible final states, in addition to a $b$-tagged jet: $e + \text{jets}$, $\mu + \text{jets}$, $e + \mu$, $e + e$ and $\mu + \mu$.
Depending on the final state, different selection criteria are applied.
These are designed to exploit the topological characteristics of the $t\overline{t}$ event.
The reconstruction is interesting in the dilepton case, where the presence of two neutrinos leads to an underconstrained decay.
This is resolved using the neutrino weighting method.

In order to determine the prefactors $\alpha_l P$ for the angular components in \eqref{parametrization}, a fit of the angular distributions of the top quark decay products is performed.
Templates for the shapes of the angular components are determined from simulation.
The parameter $C$ is fixed to the Standard Model expectation.

Interestingly, two different types of angular fits are performed: One in which the polarization inducing process is assumed to preserve $CP$ (leading to equal $\alpha_l P$) and one which is assumed to maximally violate $CP$ (leading to opposite $\alpha_l P$).

Various systematic sources of errors are investigated throughout the paper.
The most important one is the error inherent in jet reconstruction, while the second-most important one is the error inherent in simulating the $t\overline{t}$ production process to derive the signal templates.

The results for both $CP$-preserving and $CP$-violating $\alpha_l P$ are presented for all decay channels under consideration.
They are fully compatible with the Standard Model expectation of negligible polarization.

