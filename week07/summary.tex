
\twocolumn[{%
\begin{center}
  {\LARGE \textbf{\textsf{Top Quark Seminar 7}}} \\
  \vspace{1em}
  {\Large \textbf{\textsf{Igor Babuschkin}}} \\
  \vspace{1em}
  {\large \textbf{\textsf{10th February 2015}}}
  \section*{Summary of \enquote{Search for top quark decays $t\to qH$ with $H\to\gamma\gamma$ using the ATLAS detector}}
\end{center}
}]

\noindent
The paper\cite{aad} documents a search for flavour-changing neutral currents in the decay of a top quark to an up-type quark and a Higgs boson performed using data taken at the ATLAS detector corresponding to integrated luminosities of \SI{4.7}{\text{fb}^-1} at $\sqrt{s} = \SI{7}{TeV}$ and \SI{20.3}{\text{fb}^-1} at $\sqrt{s} = \SI{8}{TeV}$.

Specifically, $t\overline{t}$ events were studied where one of the top quarks decays to $Wb$ and the $W$ decays either leptonically or hadronically (which is the dominant Standard Model process) and the other top quark might decay through the previously mentioned flavour-changing neutral current involving the Higgs boson.

The basic idea of the analysis is as follows:

The selection of the $H\to\gamma \gamma$ analysis, which consists of cuts on $E_T$ as well as identification and isolation requirements for the photons is used to select $H$ candidates.
In addition, cuts are applied that select $t\overline{t}$ events.
These are chosen appropriately, depending on the type of secondary top quark decay (semi-leptonic or hadronic).

The diphoton invariant mass can then be studied to estimate the amount of signal in the data sample.

% Look at diphoton invariant mass to estimate background
% Look at both lepton and hadronic final states
% Four jets, at least one b-tagged
% combine gamma+gamma+jet and jet+jet+b/lepton+b to top quarks 
% That's all (no b-tag association or m_W constraint) -> didn't improve analysis
% 

% What are the systematic uncertainties?
% 


