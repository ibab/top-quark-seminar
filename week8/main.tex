\documentclass[
  bibliography=totoc,
  captions=tableheading,
  titlepage=firstiscover,
  twocolumn,
  %10pt,
]{scrartcl}

\setlength{\oddsidemargin}{0.0 cm}
\setlength{\evensidemargin}{0.0 cm}
\setlength{\topmargin}{-1cm}
\setlength{\textheight}{24 cm}
\setlength{\textwidth}{16 cm}

\pagestyle{plain}

\usepackage{fixltx2e}
\usepackage[aux]{rerunfilecheck}
\usepackage{polyglossia}
%\frenchspacing
\setmainlanguage[variant=british]{english}
\usepackage[intlimits]{amsmath}
\usepackage{amssymb}
\usepackage{mathtools}
\usepackage{fontspec}
\setsansfont{Helvetica}
\setmainfont{Charis SIL}
\setmonofont[Scale=0.92]{Source Code Pro}
\defaultfontfeatures{Ligatures=TeX}
\usepackage[
  math-style=ISO,
  bold-style=ISO,
  sans-style=italic,
  nabla=upright,
  partial=upright,
]{unicode-math}
\setmathfont{Tex Gyre Pagella Math}
%\setmathfont{Asana Math}
%\setmathfont{Latin Modern Math}
\setmathfont[range={\mathscr, \mathbfscr}]{XITS Math}
\setmathfont[range=\coloneq]{XITS Math}
\setmathfont[range=\propto]{XITS Math}
\removenolimits{\int}
\let\hbar\relax
\DeclareMathSymbol{\hbar}{\mathord}{AMSb}{"7E}
\DeclareMathSymbol{ℏ}{\mathord}{AMSb}{"7E}
\usepackage[autostyle]{csquotes}
\usepackage[
  locale=US,
  separate-uncertainty=true,
  per-mode=symbol-or-fraction,
]{siunitx}
\usepackage[version=3]{mhchem}
\usepackage{xfrac}
%\usepackage[section, below]{placeins}
\usepackage[
  labelfont=bf,
  font=small,
  width=0.9\textwidth,
]{caption}
\usepackage{subcaption}
\usepackage{graphicx}
\usepackage{grffile}
\usepackage{float}
\usepackage[italic]{hepnicenames}
\floatplacement{figure}{htbp}
\floatplacement{table}{htbp}
\usepackage{booktabs}
\usepackage{pdflscape}
\usepackage[
  sorting=none,
]{biblatex}
\addbibresource{main.bib}
\usepackage{microtype}
\usepackage{blindtext}
\usepackage[
  unicode,
  pdfusetitle,
  pdfcreator={},
  pdfproducer={},
]{hyperref}
\usepackage{bookmark}
\usepackage[shortcuts]{extdash}
\usepackage{tikz}
\captionsetup{width=0.45\textwidth}


\usetikzlibrary{calc}
\usetikzlibrary{arrows,shapes}

\newcommand{\package}[1]{\texttt{#1}}
\newcommand{\class}[1]{\texttt{#1}}
\newcommand{\function}[1]{\texttt{#1}}
\newcommand{\fpath}[1]{\texttt{#1}}

\captionsetup{width=0.45\textwidth}

\newcommand{\eg}{e.g.\@ }

\usepackage{hyphenat}

\begin{document}

\twocolumn[{%
\vspace{-4em}
\begin{center}
  {\LARGE \textbf{\textsf{Top Quark Seminar 8}}} \\
  \vspace{1em}
  {\Large \textbf{\textsf{Igor Babuschkin}}} \\
  \vspace{1em}
  {\large \textbf{\textsf{8th December 2014}}}
  \section*{Summary of \enquote{Lifting degeneracies in Higgs couplings using single top production in association with a Higgs boson}}
\end{center}
}]

The paper\cite{farina} presents a possible strategy to lift the ambiguity of current estimates of the Yukawa couplings of the Higgs boson to fermions.
Such an ambiguity arises in different decay channels, the most important being $\HepProcess{\Ph\to\Pgamma\Pgamma}$.
The authors propose a measurement investigating the hard process \HepProcess{\PW\Pqb\to\Pqt\Ph}, in which the Higgs coupling to fermions interferes almost completely destructively with other couplings within the Standard Model.

Such a Higgs boson production with associated single top could be studied as part of the signal $\HepProcess{\Pp\Pp\to\Pqt\Ph j}$ at the LHC.
If the Higgs boson is assumed to decay into $\Pqb\APqb$, the experimental signature would be a lepton, missing energy and three or four $\Pqb$-jets.

The authors proceed to simulate the decay and to estimate how many signal events might be observed at the LHC.
Finally, they investigate to what extent such an analysis could lift the degeneracy of the Higgs boson coupling observed in other decay channels.

The process \HepProcess{\PW\Pqb\to\Pqt\Ph} is calculated in the framework of effective field theory by factorizing the \PW radiation from the \Pqb scattering.
From the resulting amplitude, it is evident that the Higgs to fermion coupling cancels within the Standard Model ($c_V = c_F$, where $c_V$ and $c_F$ are the Higgs to vector and fermion coupling constants divided by their Standard Model expectations).
For $c_V \neq c_F$, the amplitude grows with $\sqrt{s}$ , which leads to a scattering cross section that approaches a constant value for large $\sqrt{s}$.
This clearly violates unitarity, but should be valid below a certain cutoff scale, which the authors determine to be at roughly \SI{10}{TeV}, which surpasses energies probed at the LHC.

The authors simulate signal candidates using their model and proceed to take detector effects into account by smearing and rescaling the jet $p_T$ values and applying acceptance cuts.
They further apply \Pqb-tagging and lepton reconstruction efficiencies.

Various backgrounds are generated for final states with three or four \Pqb-tags and selection cuts are applied.
From the number of remaining events (comparing the $c_F=1$ case with $c_F=-1$), the authors obtain exclusion limits on the value of $c_F$.
Combining the results for three and four final state \Pqb-jets through Fisher's method, the authors obtain exclusion limits in the $c_V$-$c_F$-plane for datasets generated with $\sqrt{s}=\SI{8}{TeV}$ and $\sqrt{s}=\SI{14}{TeV}$, corresponding to current and future center-of-mass energies at the LHC.
In order to decide how relevant their results are with respect to recent Higgs coupling measurements, they add combined results from ATLAS, CMS and Tevatron \cite{espinosa} to their exclusion plot.
The combined Higgs coupling results feature a clear degeneracy.

For $\sqrt{s}=\SI{8}{TeV}$ the authors conclude that parts of the region relevant for $c_F=-1$ can be excluded, while at $\sqrt{s}=\SI{14}{TeV}$ a conclusive removal of the degeneracy is possible, assuming integrated luminosities of \num{25} and \SI{50}{\per\femto\barn} respectively.

The authors claim that their results motivate undertaking full analyses of Higgs production with an associated top quark at ATLAS and CMS.

\vspace{1em}

\nocite{*}
\printbibliography

\end{document}
