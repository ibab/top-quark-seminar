\documentclass[
  bibliography=totoc,
  captions=tableheading,
  titlepage=firstiscover,
  twocolumn,
  %10pt,
]{scrartcl}

\setlength{\oddsidemargin}{0.0 cm}
\setlength{\evensidemargin}{0.0 cm}
\setlength{\topmargin}{-1cm}
\setlength{\textheight}{24 cm}
\setlength{\textwidth}{16 cm}

\pagestyle{plain}

\usepackage{fixltx2e}
\usepackage[aux]{rerunfilecheck}
\usepackage{polyglossia}
%\frenchspacing
\setmainlanguage[variant=british]{english}
\usepackage[intlimits]{amsmath}
\usepackage{amssymb}
\usepackage{mathtools}
\usepackage{fontspec}
\setsansfont{Helvetica}
\setmainfont{Charis SIL}
\setmonofont[Scale=0.92]{Source Code Pro}
\defaultfontfeatures{Ligatures=TeX}
\usepackage[
  math-style=ISO,
  bold-style=ISO,
  sans-style=italic,
  nabla=upright,
  partial=upright,
]{unicode-math}
\setmathfont{Tex Gyre Pagella Math}
%\setmathfont{Asana Math}
%\setmathfont{Latin Modern Math}
\setmathfont[range={\mathscr, \mathbfscr}]{XITS Math}
\setmathfont[range=\coloneq]{XITS Math}
\setmathfont[range=\propto]{XITS Math}
\removenolimits{\int}
\let\hbar\relax
\DeclareMathSymbol{\hbar}{\mathord}{AMSb}{"7E}
\DeclareMathSymbol{ℏ}{\mathord}{AMSb}{"7E}
\usepackage[autostyle]{csquotes}
\usepackage[
  locale=US,
  separate-uncertainty=true,
  per-mode=symbol-or-fraction,
]{siunitx}
\usepackage[version=3]{mhchem}
\usepackage{xfrac}
%\usepackage[section, below]{placeins}
\usepackage[
  labelfont=bf,
  font=small,
  width=0.9\textwidth,
]{caption}
\usepackage{subcaption}
\usepackage{graphicx}
\usepackage{grffile}
\usepackage{float}
\usepackage[italic]{hepnicenames}
\floatplacement{figure}{htbp}
\floatplacement{table}{htbp}
\usepackage{booktabs}
\usepackage{pdflscape}
\usepackage[
  sorting=none,
]{biblatex}
\addbibresource{main.bib}
\usepackage{microtype}
\usepackage{blindtext}
\usepackage[
  unicode,
  pdfusetitle,
  pdfcreator={},
  pdfproducer={},
]{hyperref}
\usepackage{bookmark}
\usepackage[shortcuts]{extdash}
\usepackage{tikz}
\captionsetup{width=0.45\textwidth}


\captionsetup{width=0.45\textwidth}

\begin{document}

\twocolumn[{%
\begin{center}
  {\LARGE \textbf{\textsf{Top Quark Seminar 13}}} \\
  \vspace{1em}
  {\Large \textbf{\textsf{Igor Babuschkin}}} \\
  \vspace{1em}
  {\large \textbf{\textsf{10th February 2015}}}
  \section*{Summary of \enquote{Top $B$ Physics at the LHC}}
\end{center}
}]

\noindent
The paper\cite{paper} presents a novel idea for studying $C\!P$ violation with the ATLAS and CMS experiments using top quark decays.
If a top-quark decays semi-leptonically, the charge of the lepton identifies both the charge of the top quark and the initial state of the created $b$ quark.
If the $b$ quark also decays semi-leptonically, the second lepton charge depends on the final state of the $b$ quark.

The authors present two measures of $C\!P$ violation based on asymmetries of the numbers of same-sign and opposite-sign lepton charges observed in such events:
\begin{equation}
  A^{ss}_{sl} = \frac{N^{++} - N^{--}}{N^{++} + N^{--}}
\end{equation}
and
\begin{equation}
  A^{os}_{sl} = \frac{N^{+-} - N^{-+}}{N^{+-} + N^{-+}}
\end{equation}
These could be sensitive to CP violation in $B$-$\overline{B}$ mixing and to possible new contributions to direct CP violation.

The proposed measurement could test the anomalous result for $A^b_{sl}$ by the DØ collaboration, which deviates from the Standard Model prediction by $3.8σ$.
It has been theorized that exotic sources of CP violation could explain the result.
The authors argue that these would influence the asymmetries $A^{ss}_{sl}$ and $A^{os}_{sl}$.

The asymmetries can be decomposed into several components.
Depending on possible assumption, like the absence of a sizable amount of direct CP violation, these could give access to observables like the mixing asymmetry $A^{bl}_\text{mix}$.

In order to demonstrate in how far measurements of $A^{ss}_{sl}$ and $A^{os}_{sl}$ could yield useful results, the authors have performed a sensitivity study.
This consisted of estimating $N^{\pm\pm}$ and $N^{\pm\mp}$ by taking into account the $t\overline{t}$ cross section, the LHC luminosity, various branching ratios involved in the studied processes, as well as several experimental efficiencies like the $b$-tagging, selection and charge-association efficiencies.
The latter takes into account the fact that the lepton charge cannot always be perfectly associated with the correct $b$ quark in the event.
The association can be done by reconstructing the top quark mass, or more efficiently, by using the matrix element method.

It is concluded that the sensitivity on the mixing asymmetry $A^{bl}_\text{mix}$ achievable with an integrated luminosity of $300\,\text{fb}^{-1}$ would be comparable to the current combined sensitivity of the $B$ factories.
In addition, the measurements could provide upper bounds on direct CP violation asymmetries at the permille level.

According to the authors, by extending the analysis to events with single $b$ tags and by taking into account top quarks generated from single top production, the presented studies could be improved further.

Finally, the authors claim that in addition to the presented studies, time-dependent measurements of CP violation could be conducted.
In these, asymmetries would be studied depending on the reconstructed decay time of the $b$ quark, similar to analyses performed at the $B$ factories.
A time-dependent CP violation study using top quarks could complement these measurements in a high-$p_T$ regime.

\vspace{1em}

\nocite{*}
\printbibliography

\end{document}
