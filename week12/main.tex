\documentclass[
  bibliography=totoc,
  captions=tableheading,
  titlepage=firstiscover,
  twocolumn,
  %10pt,
]{scrartcl}

\setlength{\oddsidemargin}{0.0 cm}
\setlength{\evensidemargin}{0.0 cm}
\setlength{\topmargin}{-1cm}
\setlength{\textheight}{24 cm}
\setlength{\textwidth}{16 cm}

\pagestyle{plain}

\usepackage{fixltx2e}
\usepackage[aux]{rerunfilecheck}
\usepackage{polyglossia}
%\frenchspacing
\setmainlanguage[variant=british]{english}
\usepackage[intlimits]{amsmath}
\usepackage{amssymb}
\usepackage{mathtools}
\usepackage{fontspec}
\setsansfont{Helvetica}
\setmainfont{Charis SIL}
\setmonofont[Scale=0.92]{Source Code Pro}
\defaultfontfeatures{Ligatures=TeX}
\usepackage[
  math-style=ISO,
  bold-style=ISO,
  sans-style=italic,
  nabla=upright,
  partial=upright,
]{unicode-math}
\setmathfont{Tex Gyre Pagella Math}
%\setmathfont{Asana Math}
%\setmathfont{Latin Modern Math}
\setmathfont[range={\mathscr, \mathbfscr}]{XITS Math}
\setmathfont[range=\coloneq]{XITS Math}
\setmathfont[range=\propto]{XITS Math}
\removenolimits{\int}
\let\hbar\relax
\DeclareMathSymbol{\hbar}{\mathord}{AMSb}{"7E}
\DeclareMathSymbol{ℏ}{\mathord}{AMSb}{"7E}
\usepackage[autostyle]{csquotes}
\usepackage[
  locale=US,
  separate-uncertainty=true,
  per-mode=symbol-or-fraction,
]{siunitx}
\usepackage[version=3]{mhchem}
\usepackage{xfrac}
%\usepackage[section, below]{placeins}
\usepackage[
  labelfont=bf,
  font=small,
  width=0.9\textwidth,
]{caption}
\usepackage{subcaption}
\usepackage{graphicx}
\usepackage{grffile}
\usepackage{float}
\usepackage[italic]{hepnicenames}
\floatplacement{figure}{htbp}
\floatplacement{table}{htbp}
\usepackage{booktabs}
\usepackage{pdflscape}
\usepackage[
  sorting=none,
]{biblatex}
\addbibresource{main.bib}
\usepackage{microtype}
\usepackage{blindtext}
\usepackage[
  unicode,
  pdfusetitle,
  pdfcreator={},
  pdfproducer={},
]{hyperref}
\usepackage{bookmark}
\usepackage[shortcuts]{extdash}
\usepackage{tikz}
\captionsetup{width=0.45\textwidth}


\captionsetup{width=0.45\textwidth}

\begin{document}

\twocolumn[{%
\begin{center}
  {\LARGE \textbf{\textsf{Top Quark Seminar 12}}} \\
  \vspace{1em}
  {\Large \textbf{\textsf{Igor Babuschkin}}} \\
  \vspace{1em}
  {\large \textbf{\textsf{18th January 2015}}}
  \section*{Summary of \enquote{The anti-$k_t$ jet clustering algorithm}}
\end{center}
}]

\noindent
The paper\cite{antikt} presents a new jet clustering algorithm.
Jet clustering algorithms are used to reconstruct particle jets from hits in a detector and are mainly used at hadronic colliders, such as the Tevatron and LHC.
The presented algorithm is a sequential recombination algorithm, similar to the $k_t$ and Cambridge/Aachen algorithms.

Sequential recombination algorithms work by defining a distance measure $d_{ij}$ between individual pseudojets (partially reconstructed jets) in the detector:
\begin{equation}
  d_{ij} = \min\left(k_{ti}^{2p} k_{tj}^{2p}\right) \frac{Δ_{ij}^2}{R^2}
\end{equation}
Here $Δ_{ij}$ is the distance between two pseudojets in the $φ$-$η$-plane, $R$ is a radius parameter that governs the extent of the reconstructed jets in the $φ$-$η$-plane and $k_t$ is the transversal momentum of the respective pseudojet.

Initially, all hits start out as individual pseudojets.
They are then combined by finding the minimum $d_{ij}$ and creating a new pseudojet from pseudojets $i$ and $j$.
If the distance $d_{iB}$ of a jet $i$ with respect to the beam is the minimal distance, then jet $i$ has been reconstructed and is removed from the combination process.

The algorithm behaves differently for different parameters $p$.
The case $p=1$ corresponds to the $k_t$ algorithm, while $p=0$ corresponds to the Cambridge/Aachen algorithm.
The authors explain that for $p=-1$, a new algorithm (called the \enquote{anti-$k_t$ algorithm} emerges that features several desirable properties.

First of all, like the other sequential recombination algorithms, it is collinear safe, which means it can be simulated without the occurence of cross section divergences.
Another algorithm with this property is SISCone.
The $k_t$ algorithm, the Cambridge/Aachen algorithm and SISCone all share the property of creating jets with a \enquote{soft-adaptable} boundary, which means that the jet boundaries are sensitive to fluctuations of soft radiation.

This is in contrast to the anti-$k_t$ algorithm, which features jets with a rigid boundary, similar to the iterative cone algorithm used at CMS and rigid cone algorithms like CellJet used by Pythia.
While a rigid boundary can facilitate the calibration of jets, simplify theoretical calculations and eliminate some momentum-resolution loss, which makes these algorithms useful, they are not collinear safe.

The anti-$k_t$ algorithm is interesting, because it combines the desirable properties of both classes of algorithms: It is collinear safe and produces jets with rigid boundaries.

The authors present several simulation results that demonstrate the properties of the anti-$k_t$ algorithm in comparison with established jet clustering algorithms, including an example application, a top quark reconstruction analysis.

They conclude that the anti-$k_t$ algorithm is a good alternative to previously used collinear unsafe algorithms.

\vspace{1em}

\nocite{*}
\printbibliography

\end{document}
