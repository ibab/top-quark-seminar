
\twocolumn[{%
\begin{center}
  {\LARGE \textbf{\textsf{Top Quark Seminar 14}}} \\
  \vspace{1em}
  {\Large \textbf{\textsf{Igor Babuschkin}}} \\
  \vspace{1em}
  {\large \textbf{\textsf{10th February 2015}}}
  \section*{Summary of \enquote{On the metastability of the Standard Model vacuum}}
\end{center}
}]

\noindent
The paper\cite{isidori} discusses in how far assumptions about the stability of the Standard Model (SM) electroweak vacuum can yield bounds on parameters like the Higgs boson mass $m_H$ and the top quark mass $m_t$.

These parameters influence the stability of the vacuum, because radiative corrections to the Higgs potential induced by the top quark can lead to the potential becoming unbounded from below on short length scales.

The authors present three different criteria for the stability of the SM vacuum:

The first and most stringent criterium is absolute stability.
It is pointed out that this may not be the most valid criterium, because we might very well live in an unstable universe and the instability would not be noticable, provided it is not too large.

The second, less stringent criterium is stability under thermal fluctuations in the early universe, which investigates stability under the assumption that the universe went through a phase with very high temperature (and thus high statistical fluctuation).
It relies on several cosmological assumptions for which there has been no direct experimental confirmation so far.

The third criterium, stability under quantum fluctuations without the influence of temperature, relies on almost no cosmological assumptions (just the approximate age of the universe $T_U$).
This is the criterium studied by the authors.

They state the semi-classical result for the probability that the electroweak vacuum has survived quantum fluctuations so far:
\begin{equation}
  p \approx (T_U/R)^4 \mathrm{e}^{-S_0}
\end{equation}
Here, $S_0$ is the action of the \emph{bounce}, which models the transition of the vacuum from a local to an absolute minimum, and $R$ can be interpreted as the characteristic size of the bounce.

Given a certain result for $S_0$, one can enforce $p < 1$ and thus arrive at a limit on $\lambda$, the coupling constant for the quartic component of the Higgs potential.
By studying the renormalization-group evolution of $\lambda(\mu)$ (which is influenced by $m_H$ and $m_t$), one can investigate then the Higgs potential destabilizes and convert the limit on $λ$ to a region in the $m_H$-$m_t$ plane.

The main result of the paper is a calculation of one-loop corrections to the bounce action, which allows the authors to reduce theoretical uncertainties and calculate more accurate bounds for $m_H$ and $m_t$.

The resulting plot in the $m_H$-$m_t$ plane features regions that are stable (no instability), meta-stable (no instability observed up to $T_U$) and unstable.

The authors find that for a Higgs mass of $m_H = \SI{115}{GeV/\textit{c}^2}$ (as conjectured at the time of writing) and a top mass $m_t > \SI{175\pm2}{GeV}$, their results would fall in the unstable region of the $m_H$-$m_t$ plain.

Today, we know that the Higgs boson has a mass of roughly \SI{125}{GeV}.
This puts the Standard Model into the meta-stable region calculated by the authors and means that while a destabilization of the SM vacuum cannot be excluded, the stability of the SM vacuum cannot be employed to reject the theory.

