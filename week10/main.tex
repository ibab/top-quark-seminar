\documentclass[
  bibliography=totoc,
  captions=tableheading,
  titlepage=firstiscover,
  twocolumn,
  %10pt,
]{scrartcl}

\setlength{\oddsidemargin}{0.0 cm}
\setlength{\evensidemargin}{0.0 cm}
\setlength{\topmargin}{-1cm}
\setlength{\textheight}{24 cm}
\setlength{\textwidth}{16 cm}

\pagestyle{plain}

\usepackage{fixltx2e}
\usepackage[aux]{rerunfilecheck}
\usepackage{polyglossia}
%\frenchspacing
\setmainlanguage[variant=british]{english}
\usepackage[intlimits]{amsmath}
\usepackage{amssymb}
\usepackage{mathtools}
\usepackage{fontspec}
\setsansfont{Helvetica}
\setmainfont{Charis SIL}
\setmonofont[Scale=0.92]{Source Code Pro}
\defaultfontfeatures{Ligatures=TeX}
\usepackage[
  math-style=ISO,
  bold-style=ISO,
  sans-style=italic,
  nabla=upright,
  partial=upright,
]{unicode-math}
\setmathfont{Tex Gyre Pagella Math}
%\setmathfont{Asana Math}
%\setmathfont{Latin Modern Math}
\setmathfont[range={\mathscr, \mathbfscr}]{XITS Math}
\setmathfont[range=\coloneq]{XITS Math}
\setmathfont[range=\propto]{XITS Math}
\removenolimits{\int}
\let\hbar\relax
\DeclareMathSymbol{\hbar}{\mathord}{AMSb}{"7E}
\DeclareMathSymbol{ℏ}{\mathord}{AMSb}{"7E}
\usepackage[autostyle]{csquotes}
\usepackage[
  locale=DE,
  separate-uncertainty=true,
  per-mode=symbol-or-fraction,
]{siunitx}
\usepackage[version=3]{mhchem}
\usepackage{xfrac}
%\usepackage[section, below]{placeins}
\usepackage[
  labelfont=bf,
  font=small,
  width=0.9\textwidth,
]{caption}
\usepackage{subcaption}
\usepackage{graphicx}
\usepackage{grffile}
\usepackage{float}
\usepackage[italic]{hepnicenames}
\floatplacement{figure}{htbp}
\floatplacement{table}{htbp}
\usepackage{booktabs}
\usepackage{pdflscape}
\usepackage{biblatex}
\addbibresource{main.bib}
\usepackage{microtype}
\usepackage{blindtext}
\usepackage[
  unicode,
  pdfusetitle,
  pdfcreator={},
  pdfproducer={},
]{hyperref}
\usepackage{bookmark}
\usepackage[shortcuts]{extdash}
\usepackage{tikz}



\begin{document}

\twocolumn[{%
\begin{center}
  {\LARGE \textbf{\textsf{Top Quark Seminar 10}}} \\
  \vspace{1em}
  {\Large \textbf{\textsf{Igor Babuschkin}}} \\
  \vspace{1em}
  {\large \textbf{\textsf{6th January 2015}}}
  \section*{Summary of \enquote{Search for pair and single production of new heavy quarks that decay to a $Z$ boson and a third-generation quark in $pp$ collisions at $\sqrt{s}=\SI{8}{TeV}$ with the ATLAS detector}}
\end{center}
}]

The paper\cite{atlas} is a recent analysis exploring an ATLAS dataset taken during 2012 (with $\SI{20}{fb}^{-1}$) for the presence of new, vector-like heavy quarks.
Such new quarks can have a top-like charge (referred to as $T$) or a bottom-like charge ($B$).
The analysis specifically looks for the decay modes $T\to Zt$ and $B\to Zb$.
Both pair and single production of $T$ and $B$ are investigated.

% Why vector-like quarks?
% How do we search for vector-like quarks?

The presence of a $Z$ boson is inferred by combining a pair of muons or electrons with opposite charges and checking if their invariant mass is within \SI{10}{GeV} of the known $Z$ mass.

In addition, the $t$ and $b$ quarks need to be considered:
While the presence of a $b$ quark can be detected via a $b$-jet, the decaying top quark can produce further leptons (the trilepton case) and further $b$-jets.

Both signal and background events are simulated using a variety of Monte Carlo generators.
Important background final states that needed to be modeled were $Z + $jets and $Z + q\overline{q} +$ jets, as well as $WZ$, $ZZ$, $WW$ for the trilepton case.

Variables that are useful for the selection are the $Z$ boson transverse momentum

Finally, the authors conclude that no sufficient evidence for different models of $T$ and $B$ could be observed.
Lower limits on the masses of $T$ and $B$ (depending on the respective model) could be set at roughly \SI{700}{GeV}.

\nocite{*}
\printbibliography

\end{document}
