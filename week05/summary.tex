
\twocolumn[{%
\begin{center}
  {\LARGE \textbf{\textsf{Top Quark Seminar 6}}} \\
  \vspace{1em}
  {\Large \textbf{\textsf{Igor Babuschkin}}} \\
  \vspace{1em}
  {\large \textbf{\textsf{10th February 2015}}}
  \section*{Summary of \enquote{Measurement of top quark-antiquark pair production in association with a W or Z boson in pp collisions at $\sqrt{s} = \SI{8}{TeV}$}}
\end{center}
}]

\noindent
The paper\cite{associated} presents a measurement of the cross section for top quark-antiquark production in association with an electroweak gauge boson ($W$ or $Z$) by the CMS collaboration.
The analysis uses a dataset featuring an integrated luminosity of \SI{19.5}{\text{fb}^{-1}} at $\sqrt{s} = \SI{8}{TeV}$.

Three different final states are studied: The same-sign two-lepton final state is used for determining $\sigma_{t\overline{t}W}$, while the three- and four-lepton final states are used to measure $\sigma_{t\overline{t}Z}$.

The analysis is motivated by the fact that the couplings of the top quark to $W$ and $Z$ have not been precisely measured so far and could be influenced by new physics.
The studied processes are also important background in several other analyses with an even smaller expected signal yield.

The signature of the $t\overline{t}W$ event consists of four jets, two of which can be $b$-tagged as well as two leptons (electron or muon) with equal sign:
\begin{equation*}
  pp\to t\overline{t}W\to (t\to b\ell\nu) (t\to bq\overline{q}')(W\to\ell\nu)\:.
\end{equation*}
The analysis requires the two leptons and three or more jets ($>= 1$ $b$-tag), as well as various selection criteria like requirements on $p_T$ and large hadronic activity (large $H_T$).

After applying the selection, 36 events are observed.
This is fully compatible with the Standard Model expectation of $39.7\pm3.5$ events.

In the trilepton analysis, one of the top quarks decays semi-leptonically, which means the signature consists of four jets with two $b$-tags, as well as three leptons (two of which originate from the $Z$ decay):
\begin{equation*}
  pp\to t\overline{t} Z\to (t\to b q\overline{q}')(t \to b\ell \nu)(Z \to \ell\overline{\ell}).
\end{equation*}

In this case, all four jets are required to be reconstructed and at least two $b$-tags are necessary.
In addition, two of the leptons must have the same flavor, opposite signs and their invariant mass must be close to the $Z$ mass.

The trilepton analysis observes 12 signal candidates, which is fully compatible with the total expectated number of \num{12.2\pm1.8} events.

The four-lepton analysis is concerned with the case where both top quarks decay semi-leptonically:
%\begin{equation*}
%  pp \to t\overline{t} Z\to (t\to b \ell \nu )(t \to b\ell \nu )(Z \to \ell\overline{\ell}).
%\end{equation*}
In this part of the analysis, the events passing the usual selection criteria are divided into events featuring one or two $b$-tags.
In both categories, no significant deviation from the total expected number of events can be detected.

The most important systematic uncertainties identified by the authors are the correct modelling of the lepton selection (which affects final states with more leptons more strongly) as well as the signal model (Monte Carlo simulation) and the jet energy scale and resolution.

In order to extract combined estimates for the $t\overline{t}W$ and $t\overline{t}Z$ cross sections, scans of the profile likelihood ratio have been used.

The total combined result for the inclusive $t\overline{t}V$ cross section, where $V\in\{W,Z\}$ is obtained.
It features a significance of \num{3.7} standard deviations over the background-only hypothesis.

